\documentclass[a5paper,11pt]{article} 
%PACKAGES
\usepackage[utf8]{inputenc}
\usepackage{amsmath, amssymb, amsthm}
\usepackage{mathrsfs}
\usepackage{dsfont}%Indicator Function
\usepackage{tikz}
\usepackage[font=small,labelfont={bf},textfont=it]{caption}
\usepackage[dvipsnames]{xcolor}
\definecolor{myred}{HTML}{f3003d}
\definecolor{myblue}{HTML}{1f77b4}
\definecolor{mygreen}{HTML}{2EA043}
\usepackage{hyperref}
\hypersetup{colorlinks=true, linkcolor=black, citecolor=myblue,urlcolor=myblue}
\usepackage[numbers,  sort&compress]{natbib}
%\usepackage[nottoc]{tocbibind} 
\usepackage{parskip}
\usepackage[nogroupskip, nomain, acronym]{glossaries-extra}  
\usepackage{csvsimple}
\usepackage{braket}
\usepackage[left=1.0cm,right=1.0cm,top=1.5cm,bottom=1.5cm,includeheadfoot]{geometry}

\newcommand{\mc}{Marcov chain}
\newcommand{\E}[1]{\mathbb{E}\left[#1\right]}
\newcommand{\Exy}[1]{\mathbb{E}_{x,y}\left[#1\right]}
\newcommand{\Prob}{\mathbb{P}}
\newcommand{\V}{\mathbb{V}}
\newcommand{\Vxy}[1]{\mathbb{V}_{x,y}\left[#1\right]}
\newcommand{\Pxy}[1]{\mathbb{P}_{x,y}\left(#1\right)}
\newcommand{\N}{\mathbb{N}}
%MORE
\bibliographystyle{aipnum4-2.bst}

%Theorems
\theoremstyle{plain}
\newtheorem{theorem}{Theorem}[section]
\newtheorem{proposition}[theorem]{Proposition}
\newtheorem{corollary}[theorem]{Corollary}
\newtheorem{lemma}[theorem]{Lemma}
\theoremstyle{definition}
\newtheorem{definition}[theorem]{Definition}
\newtheorem{remark}[theorem]{Remark}
\newtheorem{note}[theorem]{Note}
%% Set background color to dark and text to light
%\pagecolor{black} % Set background to black
%\color{white}    % Set text color to white
\usepackage[utf8]{inputenc} % this is needed for umlauts
\usepackage[english]{babel} % this is needed for umlauts
\usepackage[T1]{fontenc}    % this is needed for correct output of umlauts in pdf

%\glsaddall
\makeglossaries
\setabbreviationstyle[acronym]{long-short-user}
\glssetcategoryattribute{acronym}{hyper}{true}
\newacronym{sfinp}{SFINP}{Strong field Ionization of Nanoparticles}
